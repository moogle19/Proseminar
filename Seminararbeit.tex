\documentclass[a4paper,12pt]{scrartcl}
\usepackage[utf8]{inputenc}
\usepackage{ngerman}
\usepackage{setspace}
\onehalfspacing
\usepackage{geometry}
\geometry{a4paper, top=25mm, left=30mm, right=25mm, bottom=25mm}

\title{Proseminar 2013}
\author{Kevin Seidel \\ Studiengang Informatik \\ Matrikelnummer: 943147}

\begin{document}

\begin{titlepage}
\begin{center}
\vspace*{1.5cm}
\begin{Large}
\textbf{Universität Osnabrück} \\[1cm]
\end{Large}

\noindent\hrulefill
\\[1.5cm]
SEMINARARBEIT \\[1cm]
zum Proseminar \\[1cm]
\textbf{Makroökonomie} \\[1cm]
im Sommersemester 2013 \\[1.5cm]
Thema : \\[1cm]
\textbf{Banken und ihr systemisches Risiko} \\[1cm]
Referent: Prof. Dr. Valeriya Dinger \\[2cm]

\end{center}
\begin{flushleft}
Vorgelegt von: \hfill Kevin Seidel \\
\hfill 943147 \\
\hfill Falkenstraße 43 \\
\hfill 49124 Georgsmarienhütte
\end{flushleft}

\end{titlepage}

\newpage

\pagenumbering{Roman} 
\setcounter{page}{2}
\tableofcontents



\newpage
\pagenumbering{arabic} 
\setcounter{page}{1} 
\section{Einleitung}
Diese Seminararbeit untersucht die wissenschaftliche Arbeit ''A framework for assessing the systemic risk of major financial institutions'' von  Xia Huang, Hao Zhou und Haibin Zhu. 
Die Herren Huang, Zhou und Zhu stellen in ihrer Arbeit eine Möglichkeit vor, dass systemische Risiko anhand einer Kenngröße darzustellen, wodurch der Risikowert einfacher zu verstehen und leichter zu vergleichen ist.

Im speziellen wird von Ihnen das systemische Risiko des Finanzsektors untersucht. Dazu werden in einem Zeitraum von sieben Jahren, zwischen 2001 und 2008, die zwölf größten US-amerikanischen Finanzinstitute\footnote{Bank of America, Bank of New York, Bear Stearns, Citibank, Goldman Sachs, JP Morgan Chase, Lehman Brothers, Merrill Lynch, Morgan Stanley, State Street Corp., Wachovia und Wells Frago} betrachtet. 
Die Frage die sie sich dabei stellten, war, wie man das systemische Risiko am Besten bestimmen und auch anschaulich darstellen kann. Eine gute Darstellung ist in sofern hilfreich, dass man durch sie auf mögliche Ausfälle der Firmen schließen kann.
Die Autoren kamen zu dem Schluss als Kenngröße für das systemische Risiko eine theoretische Versicherungsprämie zu wählen, welche die Banken in den nächsten 12 Wochen gegen Ausfälle in Höhe von 15\% oder mehr ihrer Gesamtverbindlichkeiten absichert. 

Die Besonderheiten in dieser Arbeit sind zum einen, dass die Berechnung dieser Kenngröße nicht anhand der Bilanzen, und damit quartalsweise, geschieht, sondern in sehr kurzen Zeitabständen, da auf Echtzeitdaten, wie CDS\footnote{Credit Default Swap}-Spreads und Aktienkurse für die Berechnung zurückgegriffen wird. Des Weiteren basieren die errechneten Größen nicht auf historischen Werten, sondern sind zukunftsorientiert, da sie sich unter anderem aus den prognostizierten Ausfallwahrscheinlichkeiten und den Korrelationen der Kapitalrenditen, welche mittels der CDS-Spreads bestimmt werden, errechnen, welche ebenfalls vorwärtsgerichtet sind.

Zusätzlich dazu, untersuchen sie, wie man dieses System auf Schwachstellen durchleuchtet und diese lokalisiert. Dies ist in sofern interessant, da man sich bei bekannten Schwachstellen besser auf die möglichen Auswirkungen vorbereiten  oder diese Schwachstellen direkt eliminieren kann. 
Bei dieser Frage kommen sie zu dem Schluss, dass man die Schwachstellen am Deutlichsten mittels eines Stresstests herausstellen kann. Während dieses Stresstests wird zuerst ein Wirtschaftsmodell erstellt, auf welches dann im zweiten Schritt verschiedene Szenarien, sowohl historische als auch eigens erdachte, angewendet werden und untersucht wird, welche Auswirkungen diese Szenarien auf das Wirtschaftmodell haben. Dabei setzten sie auf ein integriertes Mikro-Makro-Modell, welches die wechselseitigen Auswirkungen zwischen dem Bankenmarkt und der restlichen Wirtschaft sehr gut abbildet.

Mittels dieses Systems kommen die Autoren zu dem Schluss, das zwischen der Korrelation der Kapitalrenditen und der Ausfallwahrscheinlichkeit ein positiver Zusammenhang besteht, was heißt, dass bei steigender Ausfallwahrscheinlichkeit die Korrelation sinkt und andersherum bei sinkendem Ausfallrisiko die Korrelation steigt. Ausserdem ist die Korrelation stark von der Fed Funds Rate\footnote{Der Zinssatz, zu welchem sich die US-amerikanischen Banken und Sparkassen untereinander Geld leihen.} und dem Term Spread\footnote{Die Differenz zwischen Langzeitzinsen und Kurzzeitzinsen.} abhängig.
\newpage

\section{Einleitung ALT}
Diese Proseminararbeit behandelt die wissenschaftliche Publikation ''A framework for assessing the systemic risk of major financial institutions'' von Xia Huang, Hao Zhou und Haibin Zhu. In dieser Arbeit wird ein System zur Bestimmung des systemischen Risikos und ein Stresstest dieses Systems, am Beispiel der zwölf größten US-amerikanischen Banken, dargelegt.

Diese Publikation ist in 3 grobe Abschnitte unterteilt. Im ersten Abschnitt wird auf das methodische Vorgehen des von den Autoren entwickelten Systems eingegangen. Diese Methodik ist wiederum in zwei Abschnitte unterteilt.

Im ersten Abschnitt der Methodik geht des um das systemische Risiko. Es wird hier als Prämie für eine Versicherung gegen finanzielle Ausfälle innerhalb der nächsten 12 Monate dargestellt. Diese Versicherungsprämie errechnet sich aus den vorhergesagten Ausfallwahrscheinlichkeiten und der Korrelation der Unternehmensrentabilität. Die Messungen sind marktbasiert und damit deutlich hochfrequenter als eine Auswertung anhand von Bilanzen, welche meist quartalsweise erscheinen. 

Der zweite Abschnitt der Methodik befasst sich mit dem Stresstest ihres Systems zur Bestimmung des systemischen Risikos. Für die Durchführung dieses Stresstests wird ein integriertes Mikro-Makro-Modell genutzt, welche nicht nur die Auswirkungen des Marktes auf das Bankwesen berücksichtigt, sondern auch die Rückwirkung des Bankwesens auf den Rest der Ökonomie. Dieser Stresstest besteht aus zwei Schritten. Im ersten Schritt wird das ökonomisches Modell dazu genutzt den Zusammenhang zwischen der Anlagenqualität und den zugrundeliegenden Faktoren zu prüfen. Der zweite Schritt besteht darin, mit Hilfe des Modells, die Auswirkungen von extremen Änderungen der zugrundeliegenden Faktoren, auf den Finanzmarkt zu bestimmen. Hierbei werden, in der Regel, historische oder hypothetische Szenarien verwendet.

Auf die verwendeten Beispieldaten wird im zweiten Abschnitt der Arbeit eingegangen. Hierbei wurden die Daten und Werte der zwölf größten amerikanischen Banken über die Jahre 2001 bis 2008 genutzt.

Im dritten und letzten Abschnitt werden die Ergebnisse dargelegt, welche durch Anwendung der Methoden aus dem ersten Teil, entstanden sind. 
\newpage

\section{Methodisches Vorgehen}
\subsection{Bestimmung des systemischen Risikos}
Zur Bestimmung des systemischen Risikos werden zwei Schritte durchlaufen. Zunächst wird das Risikoprofil eines Portfolios bestimmt. Dies geschieht anhand der Ausfallwahrscheinlichkeit und der Kapitalrenditen-Korrelation.
Im zweiten Schritt wird aus diesen beiden Werten der Indikator für das systemische Risiko errechnet, was in unserem Fall die Prämie einer theoretischen Versicherung gegen einen größeren Verlust ist.
\subsubsection{Risiko-neutrale Ausfallwahrscheinlichkeit}
Die Ausfallwahrscheinlichkeit beschreibt die Wahrscheinlichkeit, dass eine Forderung nicht zurückgezahlt wird.

In diesem Modell wird die risiko-neutrale Ausfallwahrscheinlichkeit verwendet. Bei einer risiko-neutralen Bewertung wird nicht wie sonst mit einem Zinssatz abgezinst, welcher eine Risikoprämie enthält, sondern mit dem risikofreien Zinssatz. Somit kann man den Erwartungswert berechnen und mit dem risikofreien Zinssatz auf den heutigen Wert abzinsen. Daraus erhält man den fairen Preis, welcher sowohl in der risikoneutralen, als auch in der nicht risikoneutralen Welt gilt.
Diese risiko-neutrale Ausfallwahrscheinlichkeit wird in unserem Fall direkt aus den Kreditausfall-Swap-Prämien errechnet. 

Die Formel hierzu lautet: 
\begin{equation}
PD_{i,t}=\frac{a_t s_{i,t}}{a_t LGD_{i,t}+b_t s_{i,t}}
\end{equation}
$PD_{i,t}$ ist hierbei die risiko-neutrale Ausfallwahrscheinlichkeit, $s_{i,t}$ ist die Kreditausfall-Swap-Prämie, $LGD_{i,t}$ ist die Verlustquote, $a_t \equiv \int_{t}^{t+T} e^{-r \tau}d \tau$ und $b_t \equiv \int_{t}^{t+T} \tau e^{-r \tau}d \tau$, wobei $r$ der risikofreie Zinssatz ist.

Das die Ausfallwahrscheinlichkeit aus den Kreditausfall-Swaps bestimmt werden, ist eine risiko-neutrale Messung, da nicht nur die Verlustwahrscheinlichkeit, sondern auch die Risikoprämie mit einbezogen wird. Des Weiteren sind die Werte vorwärtsgerichtet, denn die Kreditausfall-Swap-Prämien sind für eine bestimmte Vertragsperiode gegeben. 

Des Weiteren wird eine flache Terminstrukturkurve für die Ausfallintensität adaptiert. In der Realität könnte diese Annahme verletzt werden, es kommt aber nur zu geringer Beeinflussung, sodass das Gesamtergebnis nicht beeinträchtigt wird.
\newpage
\subsubsection{Kapitalrenditen-Korrelation}
Um die Kreditausfall-Korrelation zu bestimmen, existieren zwei mögliche Ansätze.
Zum einen kann man anhand von historischen Daten bezüglich der Kreditausfall die Korrelation bestimmen. Da es aber selten zu Kreditausfällen kommt sind diese Daten nicht sehr genau, vorallem im Bezug auf große Banken.
Die zweite Möglichkeit ist die Kreditausfall-Korrelation indirekt zu bestimmen. Dies geschieht mit Hilfe der Gesamtkapitalrenditen-Korrelation unter der Berücksichtigung des Aktien- und Kreditmarktes. Hull and White (2004) schlugen jedoch vor, die Gesamtkapitalrenditen-Korrelation durch die Fremdkapitalrenditen-Korrelation darzustellen. Dies wird in diesem Fall auch genau so umgesetzt. Die Gründe dafür sind zum einen, dass das Fremdkapital der am schnellsten gehandelte Anlagegegenstand ist, denn man sieht die Veränderungen am Markt und am Anlagerisiko direkt durch schwankende Aktienpreise. Der zweite Grund ist, dass nur auf dem Aktienmarkt Daten in Echtzeit gegeben sind, was es möglich macht, Korrelation über kurze Zeiträume zu bestimmen, was vorher, bei Verwendung von Tageswerten, nicht möglich war.
Durch diese hochfrequenten Analysen ist es nun möglich genauere Korrelationswerte als zuvor zu bestimmen.

Um statt der Gesamtkapital-Korrelation die Fremdkapital-Korrelation zu nutzen, muss gegeben sein, dass ein konstanter Leverage-Effekt vorliegt, da nur in diesem Fall die Korrelationen des Gesamt -und Fremdkapital gleich sind. Dies ist jedoch nur über kürzere Zeitabschnitte gegeben, daher werden hier die Korrelationen immer für Zeitabschnitte, welche kürzer als ein Quartal sein, bestimmt. Dieser Zeitabschnitt wurde anhand von Beobachtungen bestätigt. Nachdem bleibt im Zeitraum von einem Monat der Leverage-Effekt bei elf von zwölf Banken konstant, nach zwei Monaten sind es noch 10 von 12 und nach drei Monaten sieben von zwölf. Deshalb sollte der Beobachtungsabschnitt unter einem Quartal liegen.

Durch die Nutzung von Vorhersagen der Kapitalrenditen-Korrelation, statt dem Heranziehen von historischen Werten, sind die Werte konsistent mit der Vorhersage der Ausfallwahrscheinlichkeiten. Dadurch ist unser Indikator für das systemische Risiko vorwärtsgerichtet. 

\newpage
\subsection{Stresstest}

Blabla!
\newpage
\section{Daten}
Für die empirische Überprüfung der vorher beschriebenen Methoden werden Messwerte benötigt. Hier werden Werte aus dem Bankwesen benutzt, prinzipiell lassen sich mit dieser Methodik aber auch andere Portfolios untersuchen, solange sie die benötigten Vorraussetzungen, wie z.B. einen Kreditausfall-Swap, besitzen. 
Die hier verwendeten Messwerte stammen aus dem Zeitraum von Januar 2000 bis zum Mai 2008. Es wurden die, zu der Zeit, größten amerikanischen Finanzunternehmen untersucht. Zu den gehörten die Bank of America, Bank of New York, Bear Stearns, Citibank, Goldman Sachs, JP Morgan Chase, Lehman Brothers, Merrill Lynch, Morgan Stanley, State Street Corp, Wachovia und Wells Fargo. 
Für die Analysen wurden wöchentlich aktualisierte Kreditausfall-Swap-Prämien, Korrelationswerte, welche aus den täglichen Aktienkursen errechnet wurden, und makro-finanzielle Werte, die den aktuellen Zustand der Makroökonomie wiederspiegeln,  verwendet.

\newpage
\section{Empirisches Ergebnis}
\newpage
\section{Fazit}
\newpage
\section{Literaturverzeichnis}
\newpage

\section{Anhang}


\end{document}