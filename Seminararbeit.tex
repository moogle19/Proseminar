\documentclass[a4paper,12pt]{scrartcl}
\usepackage[utf8]{inputenc}
\usepackage{ngerman}
\usepackage{setspace}
\onehalfspacing
\usepackage{geometry}
\geometry{a4paper, top=25mm, left=30mm, right=25mm, bottom=25mm}

\title{Proseminar 2013}
\author{Kevin Seidel \\ Studiengang Informatik \\ Matrikelnummer: 943147}

\begin{document}

\begin{titlepage}
\begin{center}
\vspace*{1.5cm}
\begin{Large}
\textbf{Universität Osnabrück} \\[1cm]
\end{Large}

\noindent\hrulefill
\\[1.5cm]
SEMINARARBEIT \\[1cm]
zum Proseminar \\[1cm]
\textbf{Makroökonomie} \\[1cm]
im Sommersemester 2013 \\[1.5cm]
Thema : \\[1cm]
\textbf{Banken und ihr systemisches Risiko} \\[1cm]
Referent: Prof. Dr. Valeriya Dinger \\[2cm]

\end{center}
\begin{flushleft}
Vorgelegt von: \hfill Kevin Seidel \\
\hfill 943147 \\
\hfill Falkenstraße 43 \\
\hfill 49124 Georgsmarienhütte
\end{flushleft}

\end{titlepage}

\newpage

\pagenumbering{Roman} 
\setcounter{page}{2}
\tableofcontents



\newpage
\pagenumbering{arabic} 
\setcounter{page}{1} 
\section{Einleitung}
Diese Seminararbeit untersucht die wissenschaftliche Arbeit ''A framework for assessing the systemic risk of major financial institutions'' von  Xia Huang, Hao Zhou und Haibin Zhu. 
Die Herren Huang, Zhou und Zhu stellen in ihrer Arbeit eine Möglichkeit vor, dass systemische Risiko anhand einer Kenngröße darzustellen, wodurch der Risikowert einfacher zu verstehen und leichter zu vergleichen ist.

Im speziellen wird von Ihnen das systemische Risiko des Finanzsektors untersucht. Dazu werden in einem Zeitraum von sieben Jahren, zwischen 2001 und 2008, die zwölf größten US-amerikanischen Finanzinstitute\footnote{Bank of America, Bank of New York, Bear Stearns, Citibank, Goldman Sachs, JP Morgan Chase, Lehman Brothers, Merrill Lynch, Morgan Stanley, State Street Corp., Wachovia und Wells Frago.} betrachtet. 
Die Frage die sie sich dabei stellten, war, wie man das systemische Risiko am Besten bestimmen und auch anschaulich darstellen kann. Eine gute Darstellung ist in sofern hilfreich, dass man durch sie auf mögliche Ausfälle der Firmen schließen kann.
Die Autoren kamen zu dem Schluss als Kenngröße für das systemische Risiko eine theoretische Versicherungsprämie zu wählen, welche die Banken in den nächsten 12 Wochen gegen Ausfälle in Höhe von 15\% oder mehr ihrer Gesamtverbindlichkeiten absichert. 

Die Besonderheiten in dieser Arbeit sind zum einen, dass die Berechnung dieser Kenngröße nicht anhand der Bilanzen, und damit quartalsweise, geschieht, sondern in sehr kurzen Zeitabständen, da auf Echtzeitdaten, wie CDS\footnote{Credit Default Swap.}-Spreads und Aktienkurse für die Berechnung zurückgegriffen wird. Des Weiteren basieren die errechneten Größen nicht auf historischen Werten, sondern sind zukunftsorientiert, da sie sich unter anderem aus den prognostizierten Ausfallwahrscheinlichkeiten und den Korrelationen der Kapitalrenditen, welche mittels der CDS-Spreads bestimmt werden, errechnen, welche ebenfalls vorwärtsgerichtet sind.

Zusätzlich dazu, untersuchen sie, wie man dieses System auf Schwachstellen durchleuchtet und diese lokalisiert. Dies ist in sofern interessant, da man sich bei bekannten Schwachstellen besser auf die möglichen Auswirkungen vorbereiten  oder diese Schwachstellen direkt eliminieren kann. 
Bei dieser Frage kommen sie zu dem Schluss, dass man die Schwachstellen am Deutlichsten mittels eines Stresstests herausstellen kann. Während dieses Stresstests wird zuerst ein Wirtschaftsmodell erstellt, auf welches dann im zweiten Schritt verschiedene Szenarien, sowohl historische als auch eigens erdachte, angewendet werden und untersucht wird, welche Auswirkungen diese Szenarien auf das Wirtschaftmodell haben. Dabei setzten sie auf ein integriertes Mikro-Makro-Modell, welches die wechselseitigen Auswirkungen zwischen dem Bankenmarkt und der restlichen Wirtschaft sehr gut abbildet.

Mittels dieses Systems kommen die Autoren zu dem Schluss, das zwischen der Korrelation der Kapitalrenditen und der Ausfallwahrscheinlichkeit ein positiver Zusammenhang besteht, was heißt, dass bei steigender Ausfallwahrscheinlichkeit die Korrelation sinkt und andersherum bei sinkendem Ausfallrisiko die Korrelation steigt. Ausserdem ist die Korrelation stark von der Fed Funds Rate\footnote{Der Zinssatz, zu welchem sich die US-amerikanischen Banken und Sparkassen untereinander Geld leihen.} und dem Term Spread\footnote{Die Differenz zwischen Langzeitzinsen und Kurzzeitzinsen.} abhängig.

Im folgenden werde ich zuerst das Vorgehen zur Bestimmung des Indikators für das systemische Risiko, in diesem Fall die theoretische Versicherungsprämie, untersuchen und darauffolgend das System des Stresstests genauer erörtern.
\newpage

\section{Bestimmung eines Indikators für das systemische Risiko}
Um eine Kenngröße für das systemische Risiko zu ermitteln, müssen zuerst die zugrundeliegenden Wert ermittelt werden. In diesem Fall setzt sich diese aus dem Risikoprofil, der prognostizierten Ausfallwahrscheinlichkeit und der Korrelation der Vermögensrenditen zusammen. Erst nachdem diese Größen ermittelt wurden, lässt sich darauf ein Indikator für das systemisches Risiko bilden.
\subsection{Ermittlung der risikoneutralen Ausfallwahrscheinlichkeiten}
Die risikoneutrale Ausfallwahrscheinlichkeit ist eine der Hauptkomponenten bei der Bestimmung des systemischen Risikos. In dieser Arbeit wird die Ausfallwahrscheinlichkeit anhand von CDS-Spreads bestimmt.

Ein Credit Default Swap ist eine Art Verischerung für Kreditgeber. Der Kreditgeber zahlt eine Vericherungsprämie, den CDS Spread, an einen Sicherungsgeber und sichert sich damit gegen diverse Ereignisse, wie Bankrott des Kreditnehmers oder Zahlungsausfälle, ab. Die Höhe der Versicherungsprämie ist abhängig von der Ausfallwahrscheinlichtkeit und dem Verlust bei Zahlungsausfall.  Tritt ein solches Ereignisse ein, zahlt der Sicherungsgeber eine Ausgleichszahlung an der Kreditgeber.

Aus den CDS Spread lässt sich nun nach Duffie (1999) und Tarashev und Zhu (2008)\footnote{vgl. Huang et al S.5.} direkt die risikoneutrale Ausfallwahrscheinlichkeit bestimmen. Für die errechnete Ausfallwahrscheinlichkeit müssen jedoch einige Vorraussetzungen erfüllt sein. Zum einen muss eine konstante und risikofreie Zinsstruktur vorliegen, zum anderen muss eine flache Terminstruktur der Ausfallintensität gegeben sein. Des Weiteren muss eine Unabhängigkeit ziwschen dem Amortisationsrisiko und dem Ausfallrisiko existieren. 

Durch die CDS Spreads können drei Faktoren für die Ausfallwarscheinlichkeit bestimmt werden. Zuerst wäre das die Kompensation von tatsächlichen Zahlungsausfällen. Hinzu kommt die Ausfallrisikoprämie, welches eine Gebühr ist, die der Kreditnehmer an den Kreditgeber zahlt, um ihn gegen mögliche, eigene Zahlungsausfälle abzusichern. Als letztes lassen sich noch weitere Prämienzahlungen aus den CDS Spreads ableiten, wie zum Beispiel die Liquiditätsrisikoprämie, eine Rendite, welche für inliquide Anleihen gezahlt wird, da ein Kauf und Verkauf für diese Art von Anleihen schwieriger ist und dieser Mehraufwand vergütet wird.

Weitere Vorteile bei der Nutzung von CDS Spreads zur Berechnung unserer Ausfallwahrscheinlichkeit ist zum einen die Risikoneutralität. Dadurch, dass bei den CDS Spreads sowohl die wirkliche Ausfallwahrscheinlichkeit, als auch die Risikoprämien berücksichtigt werden, ist es eine risikoneutrale Messung.
Ebenso ist es ein Vorteil, das die aus CDS Spreads bestimmte Ausfallwahrscheinlichtkeit zukunftsorientiert ist. Dies ist der Fall, da durch die Verwendung von CDS Spreads, die durchschnittliche risikoneutrale Ausfallwahrscheinlichkeit innerhalb der Vertragslaufzeit des Credit Default Swaps wiedergegeben wird. 
So bekommt man nicht, wie zum Beispiel durch Bilanzen, einen Blick darauf, was mit dem Unternehmen passiert ist, sondern man bekommt einen Ausblick darauf, was in nächster Zeit mit dem Unternehmen passieren wird.
Als letztes ist anzumerken, das hier die Standardannahme einer flachen Laufzeitstruktur der Ausfallintensität gemacht wird. Diese Annahme könnte in der Realität verletzt werden, dies hat jedoch keine großen Auswirkungen auf das Endergebnis.

\subsection{Korrelation der Kapitalrenditen}
Der zweite wichtige Faktor zur Ermittlung einer Kenngröße für das systemische Risiko ist die Korrelation der Zahlungsausfälle. Um diese zu bestimmen, gibt es zwei Möglichkeiten. Zum einen kann man auf historische Daten zurückgreifen und dadurch die Korrelation der Zahlungsausfälle bestimmen. Das Problem hierbei ist, das durch die Seltenheit von Zahlungsausfälle bei großen Banken kein verlässliches Ergebnis herauskommt. 

Der zweite Ansatz besagt, dass die Ausfallkorrelation indirekt berechnet wird. Dies geschieht mit Hilfe der Korrelation der Gesamtkapitalrendite vom Aktien oder Kreditmarkt. Nach Hull und White (2004) ist es in der Praxis jedoch möglich die als Stellvertreter für die Gesamtkapitalrenditenkorrelation die Korrelation des Eigenkapitals zu verwenden. Dies ist möglich, da sich aus einer Veränderung der Aktienpreise auf die Veränderung im Gesamtkapital schließen lässt.

Dieser zweite Ansatz wird auch hier genutzt, da das Eigenkapital die höchste Liquidität auf dem Markt aufweist. Der aktuelle Zustand des Marktes und das mögliche Ausfallrisiko spiegeln sich direkt in den Aktienkursen wieder. Außerdem gibt es diese Tick-by-Tick Daten nur auf dem Aktienmarkt. 
Durch diese schnellen, über kurze Zeitintervalle bestimmten Korrelationswerte ist es möglich noch bessere Vorraussagen für die zukünftigen Korrelationen zu treffen. 

Die Benutzung der Fremdkapitalkorrelation als Ersatz für die Gesamtkapitalkorrelation ist jedoch nur bei konstantem Leverage möglich, da dann gegeben ist, das die die Korrelation des Fremdkapitals gleich der Korrelation des Gesamtkapital ist. Bei zeitlich schwankendem Leverage gehen diese Werte deutlich auseinander. 
In kurzen Zeitabständen kann man jedoch davon ausgehen, dass der Leverage relativ konstant bleibt und die Abweichungen sich nicht auf das Gesamtergebnis auswirken. Deshalb werden hier nur Zeiträume, welche kürzer als ein Quartal sind, analysiert. \footnote{Ein konstanter Leverage über einen Monat war bei 11 der 12 Banken gegeben. Nach zwei Monaten waren es noch 10 Banken. Erst ab dem Zeitraum von drei Monaten war ein großer Rückgang auf 7 Banken zu beobachten. Nach 6 Monaten wiesen nurnoch 4 Banken einen konstanten Leverage vor.}


Durch diese Herangehensweise, unter der Berücksichtigung der aktuellen Marktentwicklung, statt der Analyse von historischen Daten, ist eine vorwärtsgerichtete Untersuchung der Korrelationen möglich. Das deckt sich mit unserer Berechnung der Ausfallwahrscheinlichkeiten welche ebenfalls zukunftsorientiert sind. Aus diesen beiden Tatsachen folgt, dass auch unser Indikator für das systemische Risiko ein vorwärtsgerichteter Wert und somit zukunftsorientiert ist.

\subsection{Bestimmung des Indikators für das systemische Risiko}


\section{Methodisches Vorgehen}
\subsection{Bestimmung des systemischen Risikos}
Zur Bestimmung des systemischen Risikos werden zwei Schritte durchlaufen. Zunächst wird das Risikoprofil eines Portfolios bestimmt. Dies geschieht anhand der Ausfallwahrscheinlichkeit und der Kapitalrenditen-Korrelation.
Im zweiten Schritt wird aus diesen beiden Werten der Indikator für das systemische Risiko errechnet, was in unserem Fall die Prämie einer theoretischen Versicherung gegen einen größeren Verlust ist.
\subsubsection{Risiko-neutrale Ausfallwahrscheinlichkeit}
Die Ausfallwahrscheinlichkeit beschreibt die Wahrscheinlichkeit, dass eine Forderung nicht zurückgezahlt wird.

In diesem Modell wird die risiko-neutrale Ausfallwahrscheinlichkeit verwendet. Bei einer risiko-neutralen Bewertung wird nicht wie sonst mit einem Zinssatz abgezinst, welcher eine Risikoprämie enthält, sondern mit dem risikofreien Zinssatz. Somit kann man den Erwartungswert berechnen und mit dem risikofreien Zinssatz auf den heutigen Wert abzinsen. Daraus erhält man den fairen Preis, welcher sowohl in der risikoneutralen, als auch in der nicht risikoneutralen Welt gilt.
Diese risiko-neutrale Ausfallwahrscheinlichkeit wird in unserem Fall direkt aus den Kreditausfall-Swap-Prämien errechnet. 

Die Formel hierzu lautet: 
\begin{equation}
PD_{i,t}=\frac{a_t s_{i,t}}{a_t LGD_{i,t}+b_t s_{i,t}}
\end{equation}
$PD_{i,t}$ ist hierbei die risiko-neutrale Ausfallwahrscheinlichkeit, $s_{i,t}$ ist die Kreditausfall-Swap-Prämie, $LGD_{i,t}$ ist die Verlustquote, $a_t \equiv \int_{t}^{t+T} e^{-r \tau}d \tau$ und $b_t \equiv \int_{t}^{t+T} \tau e^{-r \tau}d \tau$, wobei $r$ der risikofreie Zinssatz ist.

Das die Ausfallwahrscheinlichkeit aus den Kreditausfall-Swaps bestimmt werden, ist eine risiko-neutrale Messung, da nicht nur die Verlustwahrscheinlichkeit, sondern auch die Risikoprämie mit einbezogen wird. Des Weiteren sind die Werte vorwärtsgerichtet, denn die Kreditausfall-Swap-Prämien sind für eine bestimmte Vertragsperiode gegeben. 

Des Weiteren wird eine flache Terminstrukturkurve für die Ausfallintensität adaptiert. In der Realität könnte diese Annahme verletzt werden, es kommt aber nur zu geringer Beeinflussung, sodass das Gesamtergebnis nicht beeinträchtigt wird.
\newpage
\subsubsection{Kapitalrenditen-Korrelation}
Um die Kreditausfall-Korrelation zu bestimmen, existieren zwei mögliche Ansätze.
Zum einen kann man anhand von historischen Daten bezüglich der Kreditausfall die Korrelation bestimmen. Da es aber selten zu Kreditausfällen kommt sind diese Daten nicht sehr genau, vorallem im Bezug auf große Banken.
Die zweite Möglichkeit ist die Kreditausfall-Korrelation indirekt zu bestimmen. Dies geschieht mit Hilfe der Gesamtkapitalrenditen-Korrelation unter der Berücksichtigung des Aktien- und Kreditmarktes. Hull and White (2004) schlugen jedoch vor, die Gesamtkapitalrenditen-Korrelation durch die Fremdkapitalrenditen-Korrelation darzustellen. Dies wird in diesem Fall auch genau so umgesetzt. Die Gründe dafür sind zum einen, dass das Fremdkapital der am schnellsten gehandelte Anlagegegenstand ist, denn man sieht die Veränderungen am Markt und am Anlagerisiko direkt durch schwankende Aktienpreise. Der zweite Grund ist, dass nur auf dem Aktienmarkt Daten in Echtzeit gegeben sind, was es möglich macht, Korrelation über kurze Zeiträume zu bestimmen, was vorher, bei Verwendung von Tageswerten, nicht möglich war.
Durch diese hochfrequenten Analysen ist es nun möglich genauere Korrelationswerte als zuvor zu bestimmen.

Um statt der Gesamtkapital-Korrelation die Fremdkapital-Korrelation zu nutzen, muss gegeben sein, dass ein konstanter Leverage-Effekt vorliegt, da nur in diesem Fall die Korrelationen des Gesamt -und Fremdkapital gleich sind. Dies ist jedoch nur über kürzere Zeitabschnitte gegeben, daher werden hier die Korrelationen immer für Zeitabschnitte, welche kürzer als ein Quartal sein, bestimmt. Dieser Zeitabschnitt wurde anhand von Beobachtungen bestätigt. Nachdem bleibt im Zeitraum von einem Monat der Leverage-Effekt bei elf von zwölf Banken konstant, nach zwei Monaten sind es noch 10 von 12 und nach drei Monaten sieben von zwölf. Deshalb sollte der Beobachtungsabschnitt unter einem Quartal liegen.

Durch die Nutzung von Vorhersagen der Kapitalrenditen-Korrelation, statt dem Heranziehen von historischen Werten, sind die Werte konsistent mit der Vorhersage der Ausfallwahrscheinlichkeiten. Dadurch ist unser Indikator für das systemische Risiko vorwärtsgerichtet. 

\newpage
\subsection{Stresstest}

Blabla!
\newpage
\section{Daten}
Für die empirische Überprüfung der vorher beschriebenen Methoden werden Messwerte benötigt. Hier werden Werte aus dem Bankwesen benutzt, prinzipiell lassen sich mit dieser Methodik aber auch andere Portfolios untersuchen, solange sie die benötigten Vorraussetzungen, wie z.B. einen Kreditausfall-Swap, besitzen. 
Die hier verwendeten Messwerte stammen aus dem Zeitraum von Januar 2000 bis zum Mai 2008. Es wurden die, zu der Zeit, größten amerikanischen Finanzunternehmen untersucht. Zu den gehörten die Bank of America, Bank of New York, Bear Stearns, Citibank, Goldman Sachs, JP Morgan Chase, Lehman Brothers, Merrill Lynch, Morgan Stanley, State Street Corp, Wachovia und Wells Fargo. 
Für die Analysen wurden wöchentlich aktualisierte Kreditausfall-Swap-Prämien, Korrelationswerte, welche aus den täglichen Aktienkursen errechnet wurden, und makro-finanzielle Werte, die den aktuellen Zustand der Makroökonomie wiederspiegeln,  verwendet.

\newpage
\section{Empirisches Ergebnis}
\newpage
\section{Fazit}
\newpage
\section{Literaturverzeichnis}

Duffie, D., 1999. Credit swap valuation. Financial Analysts Journal 55, 73-87.

Tarashev, N., Zhu, H., 2008. The pricing of portfolio credit risk: Evidence from the credit derivatives Market. Journal of Fixed Income 18, 5-24.
\newpage

\section{Anhang}


\end{document}